\documentclass[final,xcolor=svgnames]{beamer}
\usetheme{RJH}
\usepackage[orientation=portrait,size=a2,scale=1.4,debug]{beamerposter}
\usepackage[absolute,overlay]{textpos}
\setlength{\TPHorizModule}{1cm}
\setlength{\TPVertModule}{1cm}

\usepackage[T1]{fontenc}

\usefonttheme{professionalfonts}
\useinnertheme[shadow]{rounded}

\usepackage{tikz}
\usepackage{tkz-graph}
\usepackage{tkz-berge}
\usepackage{ytableau}

\renewcommand{\VertexLineWidth}{2pt}
\renewcommand{\VertexSmallMinSize}{8pt}
\renewcommand{\EdgeLineWidth}{2pt}
\GraphInit[vstyle=Hasse]


\usepackage[spanish,mexico,es-nolayout]{babel}
\usepackage[utf8]{inputenc}
\usepackage{amsmath,amscd}
\DeclareMathOperator{\sgn}{sgn}

\title{\Huge
    Representaciones del grupo simétrico en homologías}
\author{%
  Briseida Trejo Escamilla\\[15pt]
  \textsl{XXIX Coloquio Víctor Neumann-Lara de Teoría de las Gráficas, Combinatoria y sus Aplicaciones}}
\footer{\texttt{btbrisi@hotmail.com}}

\begin{document}
\begin{frame}{}
  \begin{block}{Resumen}
    \begin{minipage}{0.15\linewidth}
      \centering
      \begin{tikzpicture}[rotate=90]
        \grPetersen[RA=2,RB=1]
      \end{tikzpicture}

      $G_{5}$
    \end{minipage}
    \begin{minipage}{0.691\linewidth}
      El presente trabajo tiene como objetivo analizar la estructura de
      la homología de complejos simpliciales construidos a partir de una
      gráfica simple finita en donde actúa el grupo simétrico $S_{n}$.
      
      Se define la gráfica $M(G)$ de emparejamientos de la gráfica simple
      $G$ como la gráfica cuyos vértices son las aristas de $G$ y dos
      vértices son adyacentes si las correspondientes aristas son ajenas. En
      la gráfica $G_{n}=M(K_{n})$, el grupo~$S_{n}$ actúa de manera natural, por
      lo que sus homologías con coeficientes complejos definen
      representaciones de $S_{n}$. 
      
      La descomposición en irreducibles de tales homologías ha sido exhibida
      por Bouc (1992). Se muestra el resultado correspondiente para la gráfica de clanes $K(G_{6})$.
    \end{minipage}
    \begin{minipage}{0.15\linewidth}
      \centering
      \begin{tikzpicture}
%        \GraphInit[vstyle=Hasse]
        \grEmptyCycle[RA=0.4,rotation=-90]{5}
        \grEmptyCycle[RA=1.2,prefix=w,rotation=-90]{5}
        \grCycle[RA=2,prefix=z,rotation=90]{5}
        \EdgeInGraphMod{a}{5}{2}
        \EdgeMod{a}{w}{5}{1}
        \EdgeMod{a}{w}{5}{-1}
        \EdgeMod{w}{z}{5}{2}
        \EdgeMod{w}{z}{5}{-2}    
      \end{tikzpicture}

      $K(G_{5})$
    \end{minipage}
  \end{block}

  \vfill
  
  \begin{columns}
    \begin{column}{0.5\textwidth}
      \centering
    %   \begin{block}{Representaciones de grupos}
    %     \begin{scriptsize}
    %       \begin{itemize}         
    %       \item La \alert{representación trivial} de $S_{3}$ es el homomorfismo
    %       $\rho:S_{3}\rightarrow \mathbb{C^{*}}$ dado por $\rho(g)=1$ para todo
    %       $g\in S_{3}$.
          
        % \item Si $\pi=\tau_{1}\tau_{2}\cdots\tau_{k}$, donde $\tau_{i}$ son
        %   transposiciones, definimos la  \alert{función signo}
        %   $\sgn:\mathcal{S}_{n} \rightarrow\{\pm1\}$
        %   mediante $$\sgn(\pi):=(-1)^{k}.$$
        %   Con acción:
        %   \begin{center}
        %   $\sigma x=$
        %   \begin{equation*}
        %   \left\{
        %     \begin{array}{rl}
        %       x  & \mbox{ si $\sigma$ es par}, \\
        %       -x & \mbox{ si $\sigma$ es impar}, 
        %     \end{array}\right
        % \end{equation*}
        %   \end{center}


    %       Para $\sigma \in S_{n}$, $x\in \mathbb{C}$. Esta representación de $S_{n}$ se
    %       llama \alert{representación signo.} 
    %     \item  Sea $E=\left\{(x_{1},x_{2},x_{3})\in
    %         \mathbb{C}^{3} \mid x_{1}+x_{2}+x_{3}=0\}\right$, con
    %       acción $\sigma(x_{1},x_{2},x_{3})=(x_{\sigma(1)},x_{\sigma(2)},x_{\sigma(3)})$,
    %       esta es la \alert{representación estándar}
    %     \item  Las representaciones irreducibles de un grupo finito
    %       están en correspondencia biyectiva con las clases de
    %       conjugación de sus elementos.
    %     \end{itemize}
    %   \end{scriptsize}
    % \end{block}

      % \begin{block}{Grupo simétrico $S_{n}$}
      %   \begin{itemize}
      %   \item $(123)(45)\in S_{5}$ tiene la  \alert{estructura cíclica} $(3,2)$. 
      %   \item Dos elementos de $S_{n}$ están conjugados si y solo tienen la misma
      %     estructura cíclica.
      %   \end{itemize}
      % \end{block}
      \begin{block}{Representaciones de grupos.}
        \begin{itemize}
          \item Si $G$ es un grupo y $V$ es un espacio vectorial sobre
            $\mathbb{C}$, una \alert{representación lineal} de $G$ en
            $V$ es un homomorfismo $\rho\colon G\rightarrow GL(V).$
          \item Una representación de $G$ en $V$ da origen a una
            acción lineal de $G$ en $V$. En tal caso, decimos que $V$ es \alert{$G$-módulo.}  
         \item Un \alert{submódulo} de un $G$-módulo $V$ es un
           subespacio $U$ de $V$ tal que $gu \in U$ para todo $g\in G$ y $u\in U$.
          \item $V$ es \alert{irreducible} si los únicos submódulos de
            $V$ son $0$ y $V$.
        \end{itemize}
      \end{block}

      \begin{block}{Partición y Diagramas}
        \begin{minipage}{0.48\linewidth}
          Una \alert{partición} de $n$ es:

          A cada partición le corresponde un \alert{diagrama de
            Young}, el cual es:
          
          El número $3$ tiene tres particiones:

          Notación: $\lambda\vdash n$.
        \end{minipage}
        \begin{minipage}{0.48\linewidth}
          \ytableausetup{mathmode, boxsize=1em}
          % \begin{center}
          \begin{equation*}
            \begin{array}{cccc}
              & \lambda  &\mbox{Diagrama} \\ % \mbox{Clases de conjugación}&
              3=1+1+1   &  (1,1,1) & \ydiagram{1,1,1} \quad\\ % (1)
              \\ 3=2+1   &   (2,1)  &\ydiagram{2,1} \\ %\\(12)
              \\ 3=3     &   (3)    &\ydiagram{3} %\\(123)
            \end{array}
          \end{equation*}
          % \end{center}
        \end{minipage}
      \end{block}
     
      \begin{block}{$\lambda$-tablero}
        % \begin{scriptsize}
        Un $\lambda$-tablero es:
        
          Estos son todos los tableros correspondientes a la
          partición $\lambda=(2,1)$:
          \begin{center}
            \begin{ytableau}
              1 & 2\\
              3
            \end{ytableau} \quad
            \begin{ytableau}
              2 & 1\\
              3
            \end{ytableau}\quad
            \begin{ytableau}
              1 & 3\\
              2
            \end{ytableau}\quad
            \begin{ytableau}
              3 & 1\\
              2
            \end{ytableau}\quad
            \begin{ytableau}
              2 & 3\\
              1
            \end{ytableau}\quad
            \begin{ytableau}
              3 & 2\\
              1
            \end{ytableau}
          \end{center}
      %  \end{scriptsize}
      \end{block}

      \begin{block}{Tablero estándar}
        Un \alert{tablero estándar (Young)} es un $\lambda$-tablero cuyas
        entradas son crecientes en cada renglón y en cada columna.
        Los únicos tableros estándar para $\lambda=(2,1)$ son:
        \begin{center}
          \begin{ytableau}
            1 & 2\\
            3
          \end{ytableau}\quad
          \begin{ytableau}
            1 & 3\\
            2
          \end{ytableau}
        \end{center}
      \end{block}

      \begin{block}{Acción de $S_{n}$ en los tableros}
        Tenemos una acción de $S_{n}$ en el conjunto de
        $\lambda$-tableros $t$, donde $\lambda\vdash n$.  Dada
        $\sigma\in S_{n}$ el tablero $\sigma\cdot t$ es el tablero que
        coloca el número $\sigma(i)$ en la caja donde $t$ coloca a
        $i$. Por ejemplo:
        \begin{center}$(123)(45)$
          \ytableausetup{mathmode, boxsize=1em}
          \begin{ytableau}
            1 & 2 & 4 & 5 \\
            3 & 6\\
            7
          \end{ytableau}
          $=$
          \begin{ytableau}
            2 & 3 & 5 & 4 \\
            1 & 6\\
            7
          \end{ytableau}
        \end{center}
      \end{block}

      \begin{block}{Grupo columna $C_{t}$}
        El \alert{grupo columna} es:
        \begin{center}$t=$
          \ytableausetup{mathmode,boxsize=1em}  
          \begin{ytableau}
            4 & 1 & 2\\
            3 & 5
          \end{ytableau}\quad
          $C_{t}=S_{\{3,4\}}\times S_{\{1,5\}}\times S_{\{2\}}$
        \end{center}
      \end{block}

      \begin{block}{Politabloides}
        Si $t$ es un tablero, entonces el \alert{politabloide} asociado es
        $$e_{t}=\sum_{\pi\in C_{t}}\sgn(\pi)\pi\{t\}.$$
        \begin{center} Ejemplo:\quad $t=$
          \ytableausetup{mathmode, boxsize=1em}
          \begin{ytableau}
            1 & 2 \\
            3
          \end{ytableau}\quad
          $e_{t}=$
          \ytableausetup{mathmode, boxsize=1em,tabloids}
          \ytableaushort{12,3}
          $-$ \ytableaushort{32,1}
        \end{center}
      \end{block}

      \begin{block}{Módulo de Specht $S^{\lambda}$}
        % Para cualquier partición $\lambda$, el correspondiente
        % \alert{módulo de Specht}, denotado $S^{\lambda}$, es el
        % módulo generado por los politabloides $e_{t}$, donde $t$ es
        % tomado sobre todos los tableros de forma $\lambda$.
        Sea $\lambda$ cualquier partición. El conjunto $\{e_{t}:t
        \mbox{ es un }\lambda\mbox{-tablero estándar}\}$
        forma una base para $S^{\lambda}$ como espacio vectorial
        sobre $\mathbb{C}$.
      \end{block} 

       \begin{block}{Teorema}
       Los módulos de Specht $S^{\lambda}$ para $\lambda\vdash n$ forman
       una lista completa de representaciones irreducibles de $S_{n}$ sobre $\mathbb{C}$.
     \end{block}
    \end{column}

    \begin{column}{0.5\textwidth}
 %\begin{block}{Lema}
       % \begin{scriptsize}
        %  Sea $t$ un tablero y $\pi$ una permutación. Entonces $e_{\pi t}=\pi e_{t}$.
         %\end{scriptsize}
      %\end{block}
      
      % \begin{block}{Ejemplo}
      %   Considere $\lambda=(3)$. Hay un único politablide, es decir,
      %   \begin{center}
      %     \ytableausetup{mathmode, boxsize=1em,tabloids}    
      %     \ytableaushort{123} 
      %   \end{center}
      %   El politabloide está fijo por $S_{3}$, así que $S^{(3)}$ es la
      %   \alert{representación trivial}.
        
      %   Sea $\lambda=(1^{3})=(1,1,1)$. Sea
      %   \begin{center}$t=$
      %     \ytableausetup{mathmode, boxsize=1.5em,notabloids}
      %     \begin{ytableau}
      %       1\\
      %       2\\
      %       3
      %     \end{ytableau}
      %   \end{center}
      %   Para cualquier $\lambda$-tablero $t^{'}$,
      %   $e_{t}=e_{t^{'}}$ si $t^{'}$ es obtenida de $t$ mediante
      %   una permutación par, o $e_{t}=-e_{t^{'}}$ su $t^{'}$ es
      %   obtenida de $t$ mediante una permutación impar. Tenemos $\pi
      %   e_{t}=e_{\pi t}=\sgn(\pi)e_{t}$. Así que $S^{(1^{3})}$ es la
      %   \alert{representación signo}.
        
      %    Ahora sea $\lambda=(2,1)$, $\{t_{i}\}$ denota el $\lambda$-tabloide
      %    con $i$ en el segundo renglón, vemos que los politabloides tienen la
      %    forma  $\{t_{i}\}- \{t_{j}\}$. En efecto, el politabloide construido
      %    de el tablero
      %    \begin{center}$t=$
      %      \ytableausetup{mathmode, boxsize=1.5em,notabloids}
      %      \begin{ytableau}
      %        j & a \\
      %        i\\
      %      \end{ytableau}
      %    \end{center}
      %    es igual a $\{t_{i}\}- \{t_{j}\}$. Usemos temporalmente
      %    $\boldsymbol{e_{i}}$ para denotar el tabloide $\{t_{i}\}$. Entonces
      %    $S^{\lambda}$ es generado por los elementos de la forma
      %    $\boldsymbol{e_{i}}-\boldsymbol{e_{j}}$, y se sigue que
      %    $$S^{(2,1)}=\{c_{1}\boldsymbol{e_{1}}+c_{2}\boldsymbol{e_{2}}+c_{3}\boldsymbol{e_{3}}|c_{1}+c_{2}+c_{3}\}$$ 
      %    Esta es la \alert{representación estándar}. 
      %  \end{block}
      \begin{block}{Gráfica de emparejamiento}
        Se define la \alert{gráfica $M(G)$ de emparejamiento} de la gráfica
        simple $G$ como la gráfica cuyos vértices son las aristas de
        $G$ y dos vértices son adyacentes si las correspondientes
        aristas son ajenas. Tenemos especial interés en la gráficas
        $G_{n}=M(K_{n})$, donde $K_{n}$ es una gráfica completa de $n$ vértices.
      \end{block}


      \begin{block}{Gráfica de Clanes}
        \begin{itemize}
        \item Un \alert{clan} de una grafica $G$ es una subgráfica completa maximal. 
        \item La \alert{gráfica de clanes $K(G)$} de una grafica G es la gráfica de intersección de
          la familia de clanes de $G$.   
        \end{itemize}

      \end{block}
      
      \begin{block}{Ejemplo}
        \begin{minipage}{0.3\linewidth} 
          \centering
          \begin{tikzpicture}
            \draw[help lines] (-2,0);% grid (0,2);
            \GraphInit[vstyle=Classic]
            \Vertex[x=-2,y=0,Math,LabelOut,Lpos=180]{2}
            \Vertex[x=0,y=0,Math]{3}
            \Vertex[x=-2,y=2,Math,LabelOut,Lpos=180]{1}
            \Vertex[x=0,y=2,Math]{4}
            \Edge(1)(2)
            \Edge(1)(3)
            \Edge(1)(4)
            \Edge(2)(3)
            \Edge(2)(4)
            \Edge(3)(4)
          \end{tikzpicture}
          
          $K_{4}$
        \end{minipage}
        \begin{minipage}{0.3\linewidth}    
          \centering
          \begin{tikzpicture}
              \draw[help lines] (-2,0);% grid (0,2);
              \GraphInit[vstyle=Classic]
              \Vertex[x=-2,y=2,Math,LabelOut,Lpos=90]{12}
              \Vertex[x=-2,y=0,Math,LabelOut,Lpos=-90]{34}
              \Vertex[x=-1,y=0,Math,LabelOut,Lpos=-90]{24}
              \Vertex[x=-1,y=2,Math,LabelOut,Lpos=90]{13}
              \Vertex[x=0,y=2,Math,LabelOut,Lpos=90]{14}
              \Vertex[x=0,y=0,Math,LabelOut,Lpos=-90]{23}
              \Edge(12)(34)
              \Edge(24)(13)
              \Edge(14)(23)        
            \end{tikzpicture}

            $G_{4}$
          \end{minipage}
        \begin{minipage}{0.3\linewidth}    
          \centering
            \begin{tikzpicture}
              \draw[help lines] (-2,0);% grid (2,0);
              \GraphInit[vstyle=Classic]
              %       % \SetVertexMath
              \Vertex[x=-2,y=0,Math,LabelOut,Lpos=-90]{12,34}
              \Vertex[x=0, y=0,Math,LabelOut,Lpos=-90]{24,13}
              \Vertex[x=2, y=0,Math,LabelOut,Lpos=-90]{14,23} 
            \end{tikzpicture}

            $K(G_{4})$
                  \end{minipage}
      \end{block}

      \begin{block}{Diagrama auto-conjugado}
        Sea $\lambda=(4,2,1)$, el \alert{auto-conjugado de $\lambda$} 
        es $\lambda^{'}=(3,2,1,1)$
        \begin{center}
          \ytableausetup{notabloids} 
          $\lambda=$\ydiagram{4,2,1} \quad
          $\lambda^{'}=$\ydiagram{3,2,1,1}
        \end{center}
        $d(\lambda)$ es el número de cajas en la diagonal de el
        $\lambda$-tablero. En este caso $d(\lambda)=2$.
      \end{block}

      \begin{block}{Bouc}
          Para todo $n\geq1$ y $k\in \mathbb{Z}$,
          \begin{equation*}
            % \label{eq:6}
            \widetilde H_{k-1}(G_{n};\mathbb{C})\cong_{S_{n}}\bigoplus_{\substack{\lambda:\lambda\vdash n\\
              \lambda=\lambda^{'}\\d(\lambda)=n-2k}} S^{\lambda}.
          \end{equation*}
       \end{block}
       
        \begin{block}{Ejemplo, sea $\boldsymbol{n=3}$ y $\boldsymbol{k=1}$,}
          \ytableausetup{boxsize=0.25em} 
          \begin{equation*}
            % \label{eq:6}
            \widetilde H_{0}(G_{3};\mathbb{C})\cong_{S_{3}}\bigoplus_{\substack{\lambda:\lambda\vdash 3\\
              \lambda=\lambda^{'}\\d(\lambda)=3-2(1)=1}} S^{\lambda}=S^{(2,1)}=S^{\ydiagram{2,1}}.
          \end{equation*}
       \end{block}

       \begin{block}{Ejemplo, sea $\boldsymbol{n=5}$ y $\boldsymbol{k=1}$,}
          \begin{minipage}{0.48\linewidth}    
            \centering
           \begin{equation*}
             % \label{eq:6}
             \widetilde H_{0}(G_{5};\mathbb{C})\cong_{S_{5}}\bigoplus_{\substack{\lambda:\lambda\vdash 5\\
                 \lambda=\lambda^{'}\\d(\lambda)=5-2(1)=3}} S^{\lambda}=0.
           \end{equation*}
          \end{minipage}
          \begin{minipage}{0.5\linewidth}    
           \ytableausetup{boxsize=0.8em}            
          % \centering
           pues solo la partición $\lambda=(3,1,1)=\lambda^{'}$ del
           correspondiente diagrama \ydiagram{3,1,1}, pero
           $d(\lambda)=1$.
         \end{minipage}
       \end{block}
      
\begin{block}{$\boldsymbol{G_6}$ y $\boldsymbol{K(G_6)}$}
  %\begin{minipage}{0.15\linewidth}    
     %   \begin{center}
     %     \begin{tikzpicture}
     %       \GraphInit[vstyle=Shade]
     %        \SetVertexNoLabel
     %       \grComplete[RA=2]{5}
     %     \end{tikzpicture}
     %   \end{center}
     % \end{minipage}     
  \begin{minipage}{0.53\linewidth}    
    \centering
    \begin{tikzpicture}
      \GraphInit[vstyle=Shade]
      \Vertex[x=0,y=0,Math]{a}
      \grEmptyCycle[RA=1.3,prefix=b]{6}
      \grEmptyCycle[RA=2.6,prefix=c]{24}
      \EdgeFromOneToAll{a}{b}{}{6}
      \EdgeInGraphMod*{b}{6}{1}{0}{2}
      \EdgeInGraphMod*{c}{24}{1}{0}{2}
      \EdgeFromOneToSeq{b}{c}{1}{2}{5}
      \EdgeFromOneToSeq{b}{c}{2}{6}{9}
      \EdgeFromOneToSeq{b}{c}{3}{10}{13}
      \EdgeFromOneToSeq{b}{c}{4}{14}{17}
      \EdgeFromOneToSeq{b}{c}{5}{18}{21}
      \EdgeFromOneToSeq{b}{c}{0}{22}{23}
      \EdgeFromOneToSeq{b}{c}{0}{0}{1}
    \end{tikzpicture}

    $G_{6}$
  \end{minipage}
  \begin{minipage}{0.4\linewidth}  
    \centering   
    \begin{tikzpicture}
      \GraphInit[vstyle=Normal]
      % \SetVertexNoLabel
      \grCycle[RA=1,prefix=a,rotation=-90]{3}
      \grEmptyCycle[RA=2.6,prefix=b,rotation=-10]{12}
      \EdgeInGraphMod*{b}{12}{1}{0}{2}
      \EdgeFromOneToSeq{a}{b}{1}{0}{3}
      \EdgeFromOneToSeq{a}{b}{2}{4}{7}
      \EdgeFromOneToSeq{a}{b}{0}{8}{11}
    \end{tikzpicture}

    $K(G_{6})$
  \end{minipage}
\end{block}
\begin{block}{Bibliografía}
  % \cite{MR0225619}
  % \bibliographystyle{plain}
  % \bibliography{labiblio}
\end{block}
\end{column}
\end{columns}

  \vfill

  \begin{block}{Complejo de Cadenas para $\boldsymbol{K(G_{6})}$}
    \[
    \begin{CD}
      C_{2}(K(G_{6})) @>{\partial_{2}}>> C_{1}(K(G_{6})) @>{\partial_{1}}>> C_{0}(K(G_{6})) @>{\varepsilon}>> \mathbb{C}\\
      @VVV   @VVV   @VVV   @VVV    \\
      S^{(2,1,1,1,1)}\oplus S^{(3,1,1,1)} @>{\widehat\partial_{2}}>>
      S^{(2,1,1,1,1)}\oplus S^{(3,1,1,1)}\oplus S^{(4,2)}\oplus S^{(2,2,2)}\oplus S^{(3,2,1)} @>{\widehat\partial_{1}}>> 
      \mathbb{C} \oplus S^{(4,2)}\oplus S^{(2,2,2)} @>{\widehat \varepsilon}>>  \mathbb{C}
    \end{CD}
    \]
  \end{block}
\end{frame}
\end{document}
